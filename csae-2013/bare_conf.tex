
%% bare_conf.tex
%% V1.3
%% 2007/01/11
%% by Michael Shell
%% See:
%% http://www.michaelshell.org/
%% for current contact information.
%%
%% This is a skeleton file demonstrating the use of IEEEtran.cls
%% (requires IEEEtran.cls version 1.7 or later) with an IEEE conference paper.
%%
%% Support sites:
%% http://www.michaelshell.org/tex/ieeetran/
%% http://www.ctan.org/tex-archive/macros/latex/contrib/IEEEtran/
%% and
%% http://www.ieee.org/

%%*************************************************************************
%% Legal Notice:
%% This code is offered as-is without any warranty either expressed or
%% implied; without even the implied warranty of MERCHANTABILITY or
%% FITNESS FOR A PARTICULAR PURPOSE! 
%% User assumes all risk.
%% In no event shall IEEE or any contributor to this code be liable for
%% any damages or losses, including, but not limited to, incidental,
%% consequential, or any other damages, resulting from the use or misuse
%% of any information contained here.
%%
%% All comments are the opinions of their respective authors and are not
%% necessarily endorsed by the IEEE.
%%
%% This work is distributed under the LaTeX Project Public License (LPPL)
%% ( http://www.latex-project.org/ ) version 1.3, and may be freely used,
%% distributed and modified. A copy of the LPPL, version 1.3, is included
%% in the base LaTeX documentation of all distributions of LaTeX released
%% 2003/12/01 or later.
%% Retain all contribution notices and credits.
%% ** Modified files should be clearly indicated as such, including  **
%% ** renaming them and changing author support contact information. **
%%
%% File list of work: IEEEtran.cls, IEEEtran_HOWTO.pdf, bare_adv.tex,
%%                    bare_conf.tex, bare_jrnl.tex, bare_jrnl_compsoc.tex
%%*************************************************************************

% *** Authors should verify (and, if needed, correct) their LaTeX system  ***
% *** with the testflow diagnostic prior to trusting their LaTeX platform ***
% *** with production work. IEEE's font choices can trigger bugs that do  ***
% *** not appear when using other class files.                            ***
% The testflow support page is at:
% http://www.michaelshell.org/tex/testflow/



% Note that the a4paper option is mainly intended so that authors in
% countries using A4 can easily print to A4 and see how their papers will
% look in print - the typesetting of the document will not typically be
% affected with changes in paper size (but the bottom and side margins will).
% Use the testflow package mentioned above to verify correct handling of
% both paper sizes by the user's LaTeX system.
%
% Also note that the "draftcls" or "draftclsnofoot", not "draft", option
% should be used if it is desired that the figures are to be displayed in
% draft mode.
%
\documentclass[conference]{IEEEtran}
% Add the compsoc option for Computer Society conferences.
%
% If IEEEtran.cls has not been installed into the LaTeX system files,
% manually specify the path to it like:
% \documentclass[conference]{../sty/IEEEtran}





% Some very useful LaTeX packages include:
% (uncomment the ones you want to load)


% *** MISC UTILITY PACKAGES ***
%
%\usepackage{ifpdf}
% Heiko Oberdiek's ifpdf.sty is very useful if you need conditional
% compilation based on whether the output is pdf or dvi.
% usage:
% \ifpdf
%   % pdf code
% \else
%   % dvi code
% \fi
% The latest version of ifpdf.sty can be obtained from:
% http://www.ctan.org/tex-archive/macros/latex/contrib/oberdiek/
% Also, note that IEEEtran.cls V1.7 and later provides a builtin
% \ifCLASSINFOpdf conditional that works the same way.
% When switching from latex to pdflatex and vice-versa, the compiler may
% have to be run twice to clear warning/error messages.






% *** CITATION PACKAGES ***
%
\usepackage{cite}
% cite.sty was written by Donald Arseneau
% V1.6 and later of IEEEtran pre-defines the format of the cite.sty package
% \cite{} output to follow that of IEEE. Loading the cite package will
% result in citation numbers being automatically sorted and properly
% "compressed/ranged". e.g., [1], [9], [2], [7], [5], [6] without using
% cite.sty will become [1], [2], [5]--[7], [9] using cite.sty. cite.sty's
% \cite will automatically add leading space, if needed. Use cite.sty's
% noadjust option (cite.sty V3.8 and later) if you want to turn this off.
% cite.sty is already installed on most LaTeX systems. Be sure and use
% version 4.0 (2003-05-27) and later if using hyperref.sty. cite.sty does
% not currently provide for hyperlinked citations.
% The latest version can be obtained at:
% http://www.ctan.org/tex-archive/macros/latex/contrib/cite/
% The documentation is contained in the cite.sty file itself.





%% \usepackage[pdftex]{graphicx}
% *** GRAPHICS RELATED PACKAGES ***
%
\ifCLASSINFOpdf
  \usepackage[pdftex]{graphicx}
  % declare the path(s) where your graphic files are
  % \graphicspath{{../pdf/}{../jpeg/}}
  % and their extensions so you won't have to specify these with
  % every instance of \includegraphics
  % \DeclareGraphicsExtensions{.pdf,.jpeg,.png}
\else
  % or other class option (dvipsone, dvipdf, if not using dvips). graphicx
  % will default to the driver specified in the system graphics.cfg if no
  % driver is specified.
  % \usepackage[dvips]{graphicx}
  % declare the path(s) where your graphic files are
  % \graphicspath{{../eps/}}
  % and their extensions so you won't have to specify these with
  % every instance of \includegraphics
  % \DeclareGraphicsExtensions{.eps}
\fi
% graphicx was written by David Carlisle and Sebastian Rahtz. It is
% required if you want graphics, photos, etc. graphicx.sty is already
% installed on most LaTeX systems. The latest version and documentation can
% be obtained at: 
% http://www.ctan.org/tex-archive/macros/latex/required/graphics/
% Another good source of documentation is "Using Imported Graphics in
% LaTeX2e" by Keith Reckdahl which can be found as epslatex.ps or
% epslatex.pdf at: http://www.ctan.org/tex-archive/info/
%
% latex, and pdflatex in dvi mode, support graphics in encapsulated
% postscript (.eps) format. pdflatex in pdf mode supports graphics
% in .pdf, .jpeg, .png and .mps (metapost) formats. Users should ensure
% that all non-photo figures use a vector format (.eps, .pdf, .mps) and
% not a bitmapped formats (.jpeg, .png). IEEE frowns on bitmapped formats
% which can result in "jaggedy"/blurry rendering of lines and letters as
% well as large increases in file sizes.
%
% You can find documentation about the pdfTeX application at:
% http://www.tug.org/applications/pdftex





% *** MATH PACKAGES ***
%
%\usepackage[cmex10]{amsmath}
% A popular package from the American Mathematical Society that provides
% many useful and powerful commands for dealing with mathematics. If using
% it, be sure to load this package with the cmex10 option to ensure that
% only type 1 fonts will utilized at all point sizes. Without this option,
% it is possible that some math symbols, particularly those within
% footnotes, will be rendered in bitmap form which will result in a
% document that can not be IEEE Xplore compliant!
%
% Also, note that the amsmath package sets \interdisplaylinepenalty to 10000
% thus preventing page breaks from occurring within multiline equations. Use:
%\interdisplaylinepenalty=2500
% after loading amsmath to restore such page breaks as IEEEtran.cls normally
% does. amsmath.sty is already installed on most LaTeX systems. The latest
% version and documentation can be obtained at:
% http://www.ctan.org/tex-archive/macros/latex/required/amslatex/math/





% *** SPECIALIZED LIST PACKAGES ***
%
%\usepackage{algorithmic}
% algorithmic.sty was written by Peter Williams and Rogerio Brito.
% This package provides an algorithmic environment fo describing algorithms.
% You can use the algorithmic environment in-text or within a figure
% environment to provide for a floating algorithm. Do NOT use the algorithm
% floating environment provided by algorithm.sty (by the same authors) or
% algorithm2e.sty (by Christophe Fiorio) as IEEE does not use dedicated
% algorithm float types and packages that provide these will not provide
% correct IEEE style captions. The latest version and documentation of
% algorithmic.sty can be obtained at:
% http://www.ctan.org/tex-archive/macros/latex/contrib/algorithms/
% There is also a support site at:
% http://algorithms.berlios.de/index.html
% Also of interest may be the (relatively newer and more customizable)
% algorithmicx.sty package by Szasz Janos:
% http://www.ctan.org/tex-archive/macros/latex/contrib/algorithmicx/




% *** ALIGNMENT PACKAGES ***
%
%\usepackage{array}
% Frank Mittelbach's and David Carlisle's array.sty patches and improves
% the standard LaTeX2e array and tabular environments to provide better
% appearance and additional user controls. As the default LaTeX2e table
% generation code is lacking to the point of almost being broken with
% respect to the quality of the end results, all users are strongly
% advised to use an enhanced (at the very least that provided by array.sty)
% set of table tools. array.sty is already installed on most systems. The
% latest version and documentation can be obtained at:
% http://www.ctan.org/tex-archive/macros/latex/required/tools/


%\usepackage{mdwmath}
%\usepackage{mdwtab}
% Also highly recommended is Mark Wooding's extremely powerful MDW tools,
% especially mdwmath.sty and mdwtab.sty which are used to format equations
% and tables, respectively. The MDWtools set is already installed on most
% LaTeX systems. The lastest version and documentation is available at:
% http://www.ctan.org/tex-archive/macros/latex/contrib/mdwtools/


% IEEEtran contains the IEEEeqnarray family of commands that can be used to
% generate multiline equations as well as matrices, tables, etc., of high
% quality.


%\usepackage{eqparbox}
% Also of notable interest is Scott Pakin's eqparbox package for creating
% (automatically sized) equal width boxes - aka "natural width parboxes".
% Available at:
% http://www.ctan.org/tex-archive/macros/latex/contrib/eqparbox/





% *** SUBFIGURE PACKAGES ***
%\usepackage[tight,footnotesize]{subfigure}
% subfigure.sty was written by Steven Douglas Cochran. This package makes it
% easy to put subfigures in your figures. e.g., "Figure 1a and 1b". For IEEE
% work, it is a good idea to load it with the tight package option to reduce
% the amount of white space around the subfigures. subfigure.sty is already
% installed on most LaTeX systems. The latest version and documentation can
% be obtained at:
% http://www.ctan.org/tex-archive/obsolete/macros/latex/contrib/subfigure/
% subfigure.sty has been superceeded by subfig.sty.



%% \usepackage[caption=false]{caption}
%% \usepackage[font=footnotesize]{subfig}
% subfig.sty, also written by Steven Douglas Cochran, is the modern
% replacement for subfigure.sty. However, subfig.sty requires and
% automatically loads Axel Sommerfeldt's caption.sty which will override
% IEEEtran.cls handling of captions and this will result in nonIEEE style
% figure/table captions. To prevent this problem, be sure and preload
% caption.sty with its "caption=false" package option. This is will preserve
% IEEEtran.cls handing of captions. Version 1.3 (2005/06/28) and later 
% (recommended due to many improvements over 1.2) of subfig.sty supports
% the caption=false option directly:
\usepackage[caption=false,font=footnotesize]{subfig}
%
% The latest version and documentation can be obtained at:
% http://www.ctan.org/tex-archive/macros/latex/contrib/subfig/
% The latest version and documentation of caption.sty can be obtained at:
% http://www.ctan.org/tex-archive/macros/latex/contrib/caption/




% *** FLOAT PACKAGES ***
%
%\usepackage{fixltx2e}
% fixltx2e, the successor to the earlier fix2col.sty, was written by
% Frank Mittelbach and David Carlisle. This package corrects a few problems
% in the LaTeX2e kernel, the most notable of which is that in current
% LaTeX2e releases, the ordering of single and double column floats is not
% guaranteed to be preserved. Thus, an unpatched LaTeX2e can allow a
% single column figure to be placed prior to an earlier double column
% figure. The latest version and documentation can be found at:
% http://www.ctan.org/tex-archive/macros/latex/base/



%\usepackage{stfloats}
% stfloats.sty was written by Sigitas Tolusis. This package gives LaTeX2e
% the ability to do double column floats at the bottom of the page as well
% as the top. (e.g., "\begin{figure*}[!b]" is not normally possible in
% LaTeX2e). It also provides a command:
%\fnbelowfloat
% to enable the placement of footnotes below bottom floats (the standard
% LaTeX2e kernel puts them above bottom floats). This is an invasive package
% which rewrites many portions of the LaTeX2e float routines. It may not work
% with other packages that modify the LaTeX2e float routines. The latest
% version and documentation can be obtained at:
% http://www.ctan.org/tex-archive/macros/latex/contrib/sttools/
% Documentation is contained in the stfloats.sty comments as well as in the
% presfull.pdf file. Do not use the stfloats baselinefloat ability as IEEE
% does not allow \baselineskip to stretch. Authors submitting work to the
% IEEE should note that IEEE rarely uses double column equations and
% that authors should try to avoid such use. Do not be tempted to use the
% cuted.sty or midfloat.sty packages (also by Sigitas Tolusis) as IEEE does
% not format its papers in such ways.





% *** PDF, URL AND HYPERLINK PACKAGES ***
%
%\usepackage{url}
% url.sty was written by Donald Arseneau. It provides better support for
% handling and breaking URLs. url.sty is already installed on most LaTeX
% systems. The latest version can be obtained at:
% http://www.ctan.org/tex-archive/macros/latex/contrib/misc/
% Read the url.sty source comments for usage information. Basically,
% \url{my_url_here}.





% *** Do not adjust lengths that control margins, column widths, etc. ***
% *** Do not use packages that alter fonts (such as pslatex).         ***
% There should be no need to do such things with IEEEtran.cls V1.6 and later.
% (Unless specifically asked to do so by the journal or conference you plan
% to submit to, of course. )


% correct bad hyphenation here
%% \hyphenation{op-tical net-works semi-conduc-tor}


\begin{document}
\bibliographystyle{IEEEtran}
%
% paper title
% can use linebreaks \\ within to get better formatting as desired
%% \title{Bare Demo of IEEEtran.cls for Conferences}
\title{An abstraction for data-flow driven\\ concurrent programming}


% author names and affiliations
% use a multiple column layout for up to three different
% affiliations
\author{\IEEEauthorblockN{Xing Su}
\IEEEauthorblockA{School of Computer Science\\
National University of Defense Technology\\
Changsha, China.\\
pysuxing@gmail.com}
\and
\IEEEauthorblockN{Longfei Guo, Bin Ren, Jin Ye, Wenhua Dou}
\IEEEauthorblockA{School of Computer Science\\
National University of Defense Technology\\
Changsha, China.\\
\{lfguo, bren, jye, whdou\}@nudt.edu.cn}
%% \and
%% \IEEEauthorblockN{James Kirk\\ and Montgomery Scott}
%% \IEEEauthorblockA{Starfleet Academy\\
%% San Francisco, California 96678-2391\\
%% Telephone: (800) 555--1212\\
%% Fax: (888) 555--1212}
}

% conference papers do not typically use \thanks and this command
% is locked out in conference mode. If really needed, such as for
% the acknowledgment of grants, issue a \IEEEoverridecommandlockouts
% after \documentclass

% for over three affiliations, or if they all won't fit within the width
% of the page, use this alternative format:
% 
%\author{\IEEEauthorblockN{Michael Shell\IEEEauthorrefmark{1},
%Homer Simpson\IEEEauthorrefmark{2},
%James Kirk\IEEEauthorrefmark{3}, 
%Montgomery Scott\IEEEauthorrefmark{3} and
%Eldon Tyrell\IEEEauthorrefmark{4}}
%\IEEEauthorblockA{\IEEEauthorrefmark{1}School of Electrical and Computer Engineering\\
%Georgia Institute of Technology,
%Atlanta, Georgia 30332--0250\\ Email: see http://www.michaelshell.org/contact.html}
%\IEEEauthorblockA{\IEEEauthorrefmark{2}Twentieth Century Fox, Springfield, USA\\
%Email: homer@thesimpsons.com}
%\IEEEauthorblockA{\IEEEauthorrefmark{3}Starfleet Academy, San Francisco, California 96678-2391\\
%Telephone: (800) 555--1212, Fax: (888) 555--1212}
%\IEEEauthorblockA{\IEEEauthorrefmark{4}Tyrell Inc., 123 Replicant Street, Los Angeles, California 90210--4321}}




% use for special paper notices
%\IEEEspecialpapernotice{(Invited Paper)}




% make the title area
\maketitle


\begin{abstract}
%\boldmath
Concurrent programming has always been difficult because it's not as intuitional as sequential
programming. Programmers have to deal with subtle details such as forking threads and
synchronizing, which is tiresome and error prone . In this paper we 
design a high-level abstraction for data-flow drived concurrent programming using the functional
programming language Haskell. The abstraction provides a very clean, elegant and powerful
programming interface to the programmers so as to ease the work of concurrent programming.
Our abstract interface enables programmer to write sequential like programs while benefiting from
parallel hardware. The programs are automatically scheduled on multiple threads and run concurrently.
The performance is competitive and exceptions are handled properly.
\end{abstract}
% IEEEtran.cls defaults to using nonbold math in the Abstract.
% This preserves the distinction between vectors and scalars. However,
% if the conference you are submitting to favors bold math in the abstract,
% then you can use LaTeX's standard command \boldmath at the very start
% of the abstract to achieve this. Many IEEE journals/conferences frown on
% math in the abstract anyway.

% no keywords




% For peer review papers, you can put extra information on the cover
% page as needed:
% \ifCLASSOPTIONpeerreview
% \begin{center} \bfseries EDICS Category: 3-BBND \end{center}
% \fi
%
% For peerreview papers, this IEEEtran command inserts a page break and
% creates the second title. It will be ignored for other modes.
\IEEEpeerreviewmaketitle



%% \section{Introduction}
% no \IEEEPARstart
%% This demo file is intended to serve as a ``starter file''
%% for IEEE conference papers produced under \LaTeX\ using
%% IEEEtran.cls version 1.7 and later.
% You must have at least 2 lines in the paragraph with the drop letter
% (should never be an issue)
%% I wish you the best of success.

%% \hfill mds
 
%% \hfill January 11, 2007

%% \subsection{Subsection Heading Here}
%% Subsection text here.


%% \subsubsection{Subsubsection Heading Here}
%% Subsubsection text here.


% An example of a floating figure using the graphicx package.
% Note that \label must occur AFTER (or within) \caption.
% For figures, \caption should occur after the \includegraphics.
% Note that IEEEtran v1.7 and later has special internal code that
% is designed to preserve the operation of \label within \caption
% even when the captionsoff option is in effect. However, because
% of issues like this, it may be the safest practice to put all your
% \label just after \caption rather than within \caption{}.
%
% Reminder: the "draftcls" or "draftclsnofoot", not "draft", class
% option should be used if it is desired that the figures are to be
% displayed while in draft mode.
%
%\begin{figure}[!t]
%\centering
%\includegraphics[width=2.5in]{myfigure}
% where an .eps filename suffix will be assumed under latex, 
% and a .pdf suffix will be assumed for pdflatex; or what has been declared
% via \DeclareGraphicsExtensions.
%\caption{Simulation Results}
%\label{fig_sim}
%\end{figure}

% Note that IEEE typically puts floats only at the top, even when this
% results in a large percentage of a column being occupied by floats.


% An example of a double column floating figure using two subfigures.
% (The subfig.sty package must be loaded for this to work.)
% The subfigure \label commands are set within each subfloat command, the
% \label for the overall figure must come after \caption.
% \hfil must be used as a separator to get equal spacing.
% The subfigure.sty package works much the same way, except \subfigure is
% used instead of \subfloat.
%
%\begin{figure*}[!t]
%\centerline{\subfloat[Case I]\includegraphics[width=2.5in]{subfigcase1}%
%\label{fig_first_case}}
%\hfil
%\subfloat[Case II]{\includegraphics[width=2.5in]{subfigcase2}%
%\label{fig_second_case}}}
%\caption{Simulation results}
%\label{fig_sim}
%\end{figure*}
%
% Note that often IEEE papers with subfigures do not employ subfigure
% captions (using the optional argument to \subfloat), but instead will
% reference/describe all of them (a), (b), etc., within the main caption.


% An example of a floating table. Note that, for IEEE style tables, the 
% \caption command should come BEFORE the table. Table text will default to
% \footnotesize as IEEE normally uses this smaller font for tables.
% The \label must come after \caption as always.
%
%\begin{table}[!t]
%% increase table row spacing, adjust to taste
%\renewcommand{\arraystretch}{1.3}
% if using array.sty, it might be a good idea to tweak the value of
% \extrarowheight as needed to properly center the text within the cells
%\caption{An Example of a Table}
%\label{table_example}
%\centering
%% Some packages, such as MDW tools, offer better commands for making tables
%% than the plain LaTeX2e tabular which is used here.
%\begin{tabular}{|c||c|}
%\hline
%One & Two\\
%\hline
%Three & Four\\
%\hline
%\end{tabular}
%\end{table}


% Note that IEEE does not put floats in the very first column - or typically
% anywhere on the first page for that matter. Also, in-text middle ("here")
% positioning is not used. Most IEEE journals/conferences use top floats
% exclusively. Note that, LaTeX2e, unlike IEEE journals/conferences, places
% footnotes above bottom floats. This can be corrected via the \fnbelowfloat
% command of the stfloats package.

\section{Introduction}\label{sec:introduction}
Processors are not getting faster, instead the manufactors are providing us more and
more processors on a single chip. So concurrent programming is becoming the mainstream
programming technique. But concurrent programming has always been hard work. The logic
of concurrent programs are not as intuitional as the traditional sequential ones.
And programmers have to deal with lots of subtles details like forking threads,
synchronizing, catching exceptions etc.

People have been working to estimate the difficulty of concurrent programming. Some use
the Data Parallel techniques, which is adopted to vector processors and DSPs, and
works well in data oriented applications. Some trys to enhance the compiler to parallelise
the workload automatically which, however, is hark work. Some other methods need
the programmer give hints to the compiler in the code to explicitly figure out the parallel
part of the programs, for instance, CUDA programming with Nvidia GPUs.

We argue that a high-level abstraction of concurrent programming, which enforces the compiler
or library to do the hard work, can help wipe the subtle details out from the insight
of programmers. We designed an abstract interface for concurrent programmers
using the functional programming language Haskell,
and it simplify the work of concurrent programming a lot, especially for the data-flow
drived problems.

In the following sections, we will show our design and its applications. In Section 2 we
describe the Data-flow drived problems and give a concrete example to work with in the
whole paper. In Section 3  afer a brief introduction to the Haskell language and
the concept of arrows which inspired our design, we will show our main design and
the implimentation. We enhance our design
with the ablility to deal with exceptions in Section 4. In Section 5 a simple test
application is shown. Conclusion comes last in Section 5, discussing the related work, advantages
and disadvantages of our approach and future work we will do.

\section{Data-flow Drived Problems}\label{sec:dataflow-drived-problems}
Many real-world problems can be abstracted to Data-flow problems. A Data-flow problem
can be expressed as a data-flow graph, in which each node represents a task, and each
directed edge represents some kind of dependency between tasks it connects, for instance,
data dependency or timing dependency.

To give a concrete example, consider the following problem. Suppose the family is going to hold
a party for Christmas Eve and everyone has got some work to do. Dad is responsible for
purchasing all raw materials and all other work need these materials. When the materials
are available, Mum will go to prepare dinner for the family. The three kids are also sent off
to do some work. Tommy will trim the Christmas tree, David will sweep the house, and little Lucy
will make some decorators for the house. When all the kids finish their work Dad will decorate the
house. The party will begin after the house is decorated and Mum has the
dinner prepared.

This problem can be expressed by a data-flow graph, as shown in figure \ref{fig:christmas-dataflow}.
From the figure we can see that some work should be done in parallel. Mum can cook dinner while
the kids are doing there work. If we are writing a program to simulate the whole preparation work
for Christmas Eve party, we should better fork a thread for each of the parallelisable work, thus
making the simulation running more faster than a sequential one.
\begin{figure*}[!t]
  \centering
  \includegraphics[scale=0.4]{christmas-dataflow.png}
  \caption{The Chistmas Eve party data-flow}
  \label{fig:christmas-dataflow}
\end{figure*}
\begin{figure*}
  \begin{verbatim}
    id :: a b b                                    -- the identity arrow
    >>> :: a b c -> a c d -> a b d                 -- arrow composition
    arr :: (b -> c) -> a b c                       -- lift a general function to arrow
    (***) :: a b c -> a b' c' -> a (b, b') (c, c') -- combine two parallel arrows
    (&&&) :: a b c -> a b c' -> a b (c, c')        -- combine two arrows' result
  \end{verbatim}
  \caption{Combinators for arrow programming}
  \label{fig:arrows-interface}
\end{figure*}

\section{A High-level Abstraction}\label{sec:a-high-level-abstraction}
In this section, we will demonstrate our high-level abstraction for concurrent programming. First we
should give a brief introduction to the functional programming language Haskell. Then a solution of
the Christmas Eve problem using our abstraction interface will be shown, with a comparation to ordinary
programming technique. At last the implementation detail is described.

\subsection{Haskell Language and Arrows}\label{subsec:haskell-and-arrows}
\subsubsection{The Haskell Language}
The Haskell language is originally designed as a base language for researchers to explore new language
features. It belongs to the functional language family, in which there are some other widely known
languages such as Lisp and ML. 

As a functional language, functions are first-class citizens in the language, that is, functions can be
passed as parameters, returned as a return value, or manipulated in any way. There can also be
anonymous functions. Haskell is a strong static typed language and everything has an associated type.
General functions in haskell has a type signature as below.
The type signature means that function \texttt{f} takes a parameter of
type \texttt{a} and return a value of type \texttt{b}.
\begin{verbatim}
            f :: a -> b
\end{verbatim}

\subsubsection{Arrows}
Haskell also provide a kind of generalised function, called Arrow \cite{Hughes2005}.
As the name suggests, it is
a directed arrow start from some type and end at some type. An arrow of the below type signature
means that it's take a computation of a value of type \texttt{a} and produce a computation of a value
of type \texttt{b}. Note an arrow take a computation of some type, not a value, though
the computation is related to the type. So is the returned computation. The computation is specified
by the definition of arrow type \texttt{SomeArrow}.
\begin{verbatim}
      arrow :: SomeArrow a b
\end{verbatim}

Arrows provide an elegant interface for programmers. The main combinator for arrows are listed in figure
\ref{fig:arrows-interface}. \texttt{id} is the identity arrow which output its input without any processing.
\texttt{>>>} connect tow arrow of type \texttt{A a b} and \texttt{A b c} and return a new arrow of type
\texttt{A a c}. \texttt{***} combines two arrows as one without affect each other. \texttt{\&\&\&} combines
two arrows' result.

We will see that the arrow interface is used to model parallel tasks in the following subsections.

\subsection{Abstraction of Data-flow Concurrent Programming}\label{subsec:abstraction-of-dataflow}
Recall the Christmas Eve problem in Section \ref{sec:dataflow-drived-problems}
and it's data-flow expression in figure \ref{fig:christmas-dataflow}.
We see that each task in the data-flow graph is an arrow! A task
may take some input from other task's output as well as produce some output which may be the
input of other tasks.

Translate the data-flow graph into code using our abstract interface requires little effort. See
the code shown in figure \ref{fig:code-christmas}. Some functions' type signatures are omitted for
the sake of space. Compared to ordinary concurrent code gluted with \texttt{fork}s and synchronous
code, it seems to be a sequential program! In fact, a sequential version can also be coded using
arrow interface and it looks exactly like our concurrent version in figure \ref{fig:code-christmas}.

\texttt{CIO} is the arrow type we use to represent a parallel task. We use
\texttt{()} to indicate \texttt{NULL}, the arrow \texttt{buyMaterials} of type \texttt{CIO () Materials}
means a task that need no input and produce a \texttt{Materials}. Note that all tasks
represented by a \texttt{CIO} arrow will be done concurrently!
\begin{figure*}
  \begin{verbatim}
buyMaterials :: CIO () Materials
...     -- some type signature omitted for the sake of space
prepareChrist :: CIO () Party
prepareChrist = 
  buyMaterials >>>
  (id &&& ((trimTree &&& sweepHouse &&& makeDecorators) >>> decorateHouse)) >>>
  holdParty
  \end{verbatim}
  \caption{Code of Christmas Eve problem}
  \label{fig:code-christmas}
%% trimTree :: CIO Materials Tree
%% sweepHouse :: CIO Materials ()
%% makeDecorators :: CIO Materials Decorators
%% decorateHouse :: CIO (Tree, (), Decorators) ()
%% holdParty :: CIO (Dinner, ()) Party
\end{figure*}

To be exactly, an arrow \texttt{CIO () Materials} does not produce a value of \texttt{Materials},
but a promise of a value of type \texttt{Materials}, that's where the parallelisation of tasks lies.
Both the input and output of an arrow of type \texttt{CIO a b} is not an actual value, but a promise,
or future instead. You can view a promise as a box to be filled in and it's promised to be
filled in some value at some time in the future. So generally, an \texttt{CIO} task arrow
take a promise from
which it will take its input, and return a promise from which other tasks can take their input.

\subsection{Implementation}\label{subsec:implementation}
In this subsection, we will show implementation of our concurrent arrow interface. For the sake
of space, only the skeleton of implementation should be listed here. Readers who are interested in more
details can refer to the project's homepage.\cite{CIOHomepage}

\subsubsection{The Promise}
Indeed, a promise is a synchronous variable between threads. For the simplest case, we can use \texttt{MVar}
provided by Concurrent Haskell\cite{Jones1996} to implement the promise.
A variable of type \texttt{MVar a} is a box can be
taken a value of type \texttt{a} from or put a value into by threads. If a \texttt{MVar} is empty,
the threads which want to take a value from it will be put to sleep, and waken up again if some other
thread has put a value into it. Similarly, if a \texttt{MVar} contains a value, then threads that want
to put a value into it will be suspended and waken up when the \texttt{MVar} is empty again. That's
exactly what a promise does!

\subsubsection{The Concurrent Arrow \texttt{CIO}}
Given the promise implementation above, we can now define our concurrent arrow \texttt{CIO}.

First we define a task \texttt{Pool} in which we store all the concurrent tasks. A \texttt{Task} is made up with
an input promise, an output promise, and an action to perform. A task pool contains a list of tasks,
and a list of \texttt{ThreadId}s, each identify a thread we fork to do a task. The \texttt{ThreadId}s
will be used for exception handling, which will be described in section \ref{sec:exception-safety}.
After all tasks has been
added to the pool using \texttt{pushWork}, we call \texttt{runPool} to create a thread for each task to
do the work.
\begin{verbatim}
data Task = 
  forall a.
  Action (MVar a) (MVar b) (a -> IO b)
data Pool = Pool {
  pool :: IORef [Task],
  tids :: IORef [ThreadId]
}
pushWork :: Pool -> Task -> IO ()
runPool :: Pool -> IO ()
runPool p = do
  ts <- ...   -- get tasks from p
  forM ts $ \t -> fork t -- fork a thread
\end{verbatim}

Then comes the definition of the concurrent arrow. The \texttt{CIO} arrow contains a wrapped function, named as
\texttt{runC}. \texttt{runC} takes a task pool and an input promise, add a task waiting for the input promise
to the pool, and return the output promise of the new task. The output promise will be used as the input
promise for next task.
\begin{verbatim}
newtype CIO a b = {
  runC :: Pool -> MVar a -> IO (MVar b)
}
\end{verbatim}

Now we have the interface of concurrent arrow, using the arrow combiantors mentioned in section
\ref{subsec:haskell-and-arrows}, we can express data-flow problems easily. Here we show two other
examples to illustrate how to translate data-flow graphs to code using our arrow interface. The data-flow
graphs shown in figure \ref{fig:examples} can be translated as below.
\begin{verbatim}
(a) (f &&& g) >>>
    arr (\(a, b) -> ((a, b) b)) >>>
    (h *** j) >>> k
(b) (f &&& g) >>>
    arr (\(a, b) -> (a, ((a, b), b))) >>>
    (h *** (j *** id)) >>> k
\end{verbatim}
\begin{figure}
  \centerline{
    \subfloat[]{
      \includegraphics[scale=0.5]{example1.png}%
      \label{subfig:example1}
    }
    \hfil
    \subfloat[]{
      \includegraphics[scale=0.5]{example2.png}%
      \label{subfig:example2}
    }
  }
  \caption{Data-flow problem examples}
  \label{fig:examples}
\end{figure}

\subsubsection{Create Concurrent Tasks}
At last we supply some combinators to create concurrent tasks. Ordinary tasks can be
expressed by Haskell functions of type \texttt{a -> IO b}, these functions can be
converted to concurrent tasks of type \texttt{CIO a b}. One of there functions is \texttt{conc}:
\begin{verbatim}
    conc :: (a -> IO b) -> CIO a b
\end{verbatim}
Using the \texttt{conc} combinator, the data-flow problems would be coded in a programming
paradigm which looks like:
\begin{verbatim}
task = conc task1 >>>
       (conc task2 &&& conc task3) >>>
       (conc task4 *** conc task5)
\end{verbatim}
%% \begin{figure*}
%%   \begin{verbatim}
%%     conc :: (a -> IO b) -> CIO a b
%%   \end{verbatim}
%%   \caption{Code of Christmas Eve party problem}
%%   \label{fig:code-christmas}
%% \end{figure*}

\section{Exception Safety}\label{sec:exception-safety}
In practical programming, catching and handling exceptions is an important issue, especially
in a concurrent setting.\cite{Marlow2001} Enhance our design with the ability to handle exceptions is not hard
work.

Only the definition of \texttt{runPool} needs to be changed. Compare the code below to previous
one defined in section \ref{subsec:implementation}, we can see that we wrap the task with a
catch operation. If an exception occurs during running the task, the whole concurrent tasks
will be cancelled and all other running threads are killed, thus avoiding any resource leak and
no unnecessary work will be done.
\begin{verbatim}
runPool p = do
  ts <- ...   -- get tasks from p
  forM ts $ \t -> -- fork a thread
    fork (catch t handler)
\end{verbatim}

\section{Performance}
We still use the Christmas Eve problem to measure the performance of our concurrent
arrow interface. Suppose Dad will spend 0.5 hour to buy all materials, and it takes
1 hour for Mum to cook the dinner. Three kids can finish there work in 0.6, 0.8 and 0.7 hour
respectively and Dad will spend 0.3 hour decorating the house. If all work are done
sequentially, it will cost $0.5+1+0.6+0.8+0.7+0.3=3.9$ hours. But all family members should
work independently and only 1.6 hours is needed indeed.

We have three versions of implementation, the first in ordinary sequential style,
the second in ordinary concurrent style which explicitly fork threads and synchronize,
the third coded with our concurrent arrow interface.
The results of all three programs are shown in table \ref{tab:simulating-result}.
In the simulation, two OS threads are harnessed in the second and third versions.
We use one second to represent an hour. Elapsed time is measured using standard unix
\texttt{time} utility. We can see all three programs have the performance as
expected, and the version using arrow interface runs as fast as ordinary concurrent
version.
\begin{table}
%% increase table row spacing, adjust to taste
%% \renewcommand{\arraystretch}{1.3}
%% if using array.sty, it might be a good idea to tweak the value of
%% \extrarowheight as needed to properly center the text within the cells
\caption{Simulating result of Christmas Eve properly}
\label{tab:simulating-result}
\centering
\begin{tabular}{|c||c||c||c|}
\hline
Version & Sequential & Explicit Concurrency & Arrow Interface\\
\hline
Time(s) & 3.909 & 1.606 & 1.606\\
\hline
\end{tabular}
\end{table}
%% The terminal output is modified to save space. The result show that our arrow
%% interface works well and can reach almost the ideal performance.
%% \begin{verbatim}
%% $> time christmas_cio
%% "Dad: All materials are here!"
%% "Tommy: A beauful Christmas Tree!"
%% "Lucy: How beautiful the decorators are!"
%% "David: Sweeping the house is tiresome!"
%% "Mum: Dinner is ready!"
%% "Dad: The house is decorated!"
%% "All work is done! The party begins!"

%% real	0m1.606s
%% $> time christmas
%% "Dad: All materials are here!"
%% ... # same as above with order changed
%% real	0m3.909s
%% \end{verbatim}

In the Christmas Eve testcase, we put threads to sleep to simulate time elapsing and the
processors actually do no meaningful work. We will use another testcase to show how are the concurrent
tasks are scheduled on multiple OS threads. In this testcase we want to find all prime numbers in a range 
$[0, n]$. Because the primeness judgements of integral numbers are indepedent of each other, so we can
adopt a divide-and-conquer algorithm. The big range is splited to some smaller ones and computed
indepedently, and the results are collected as a whole at last. We show the data-flow graph in figure
\ref{fig:divide-and-conquer}. We use the ThreadScope utility to show the runtime statistics of our program, as
shown in figure \ref{fig:runtime-statistics}. Two OS threads are harnessed, named as HEC0 and HEC1,
the green bar means the OS thread is working, and yellow bar stands for garbage collecting operations done
by the Haskell Runtime System. From figure \ref{fig:runtime-statistics} we see that all workload are banlanced
well on multiple OS threads.
\begin{figure}
  \centering
  \includegraphics[scale=0.38]{example3.png}
  \caption{Data-flow graph of divide-and-conquer}
  \label{fig:divide-and-conquer}
\end{figure}
\begin{figure*}[!t]
  \centerline{
    \subfloat[Workload on multiple OS threads]{
      \includegraphics[scale=0.3]{divide-and-conquer.png}%
      \label{subfig:runtime-statistics}
    }
    \hfil
    \subfloat[Time line zoomed in for details]{
      \includegraphics[scale=0.3]{divide-and-conquer-details.png}%
      \label{subfig:runtime-statistics-details}
    }
  }
  \caption{Runtime statistics of divide-and-conquer}
  \label{fig:runtime-statistics}
\end{figure*}

%% \begin{figure*}
%%   \centering
%%   \includegraphics[scale=0.35]{divide-and-conquer.png}
%%   \caption{Runtime statistics of divide-and-conquer}
%%   \label{fig:runtime-statistics}
%% \end{figure*}

\section{Conclusion}
In this paper we show our design of a high-level abstraction of data-flow drived concurrent
programming, which enable the programmers to directly translate a data-flow problem into proper
code. The code looks like a sequential one, but work concurrently indeed. All labour required is
the use of our \texttt{conc} combinator to convert ordinary tasks to a concurrent ones, which means
the programmers are freed from taking care of all the subtle details of concurrent programming.

The main contributions are:
\begin{itemize}
  \item An elegant arrow interface for expressing data-flow problems.
  \item The concurrent tasks is automatically scheduled, without disturbing the programmers.
  \item Good performance with exception safety.
\end{itemize}

The idea of maintaining a task pool is inspired by Marlow's Par monad.\cite{Marlow2011a}
The Par monad works for
determinated computation, our design works with non-determinated computation.
Our work is based on Peyton Jones's Concurrent Haskell\cite{Jones1996} 
is more general than our arrow interface while our design are best fit for data-flow problems.

The disadvantage of our work is that it can only handle tasks with one parameter, say functions of
type \texttt{a -> IO b}. For multi-parameter tasks of type \texttt{a -> b -> IO c} you must uncurry
it to a task of type \texttt{(a, b) -> IO c} before calling \texttt{conc}. This is a future work
we shall consider about, And we think the using HList\cite{Kiselyov2004} and other
type-level programming techniques seems to be a promising method.

% conference papers do not normally have an appendix


% use section* for acknowledgement
\section*{Acknowledgment}
This research is supported by National Natural Science Foundation of China
(Grant No. 61103010, 61103190, 60803100), National Basic Research Program
of China (Grant No. 2012CB933500) and High Technology Research and Development
Program of China (Grant No. 2012AA012201,2012AA011902).

% trigger a \newpage just before the given reference
% number - used to balance the columns on the last page
% adjust value as needed - may need to be readjusted if
% the document is modified later
%\IEEEtriggeratref{8}
% The "triggered" command can be changed if desired:
%\IEEEtriggercmd{\enlargethispage{-5in}}

% references section

% can use a bibliography generated by BibTeX as a .bbl file
% BibTeX documentation can be easily obtained at:
% http://www.ctan.org/tex-archive/biblio/bibtex/contrib/doc/
% The IEEEtran BibTeX style support page is at:
% http://www.michaelshell.org/tex/ieeetran/bibtex/
%\bibliographystyle{IEEEtran}
% argument is your BibTeX string definitions and bibliography database(s)
%\bibliography{IEEEabrv,../bib/paper}
%
% <OR> manually copy in the resultant .bbl file
% set second argument of \begin to the number of references
% (used to reserve space for the reference number labels box)
%% \begin{thebibliography}{1}

%% \bibitem{IEEEhowto:kopka}
%% H.~Kopka and P.~W. Daly, \emph{A Guide to \LaTeX}, 3rd~ed.\hskip 1em plus
%%   0.5em minus 0.4em\relax Harlow, England: Addison-Wesley, 1999.

%% \end{thebibliography}

\bibliography{refs.bib}


% that's all folks
\end{document}


